%% fancy header & foot
\pagestyle{fancy}
\lhead{[ELEC-H-2001] Électricité\\ LABO \no 2 : Séance 2~: Les circuits réactifs en transitoire \ifthenelse{\boolean{corrige}}{~-- corrigé}{}}
\rhead{v1.0.0\\ page \thepage}
\cfoot{}
%%
\usepackage{placeins}
\pdfinfo{
/Author (Renaud Theunissen & Youssef Agram, ULB -- BEAMS-EE)
/Title (Laboratoire 2 ELEC-H-2001, Séance 2~: Les circuits réactifs en transitoire)
/ModDate (D:\pdfdate)
}

\hypersetup{
pdftitle={Labo 2 [ELEC-H-2001] Électricité : LABO 2~: Les circuits réactifs en transitoire},
pdfauthor={Renaud Theunissen & Youssef Agram, ©2020 ULB - BEAMS-EE},
}


\setlength{\parskip}{0.5cm plus4mm minus3mm} %espacement entre §
\setlength{\parindent}{0pt}



\begin{document}

\tptitle{}{Séance 2~: Les circuits réactifs en transitoire}
Le but de cette séance est de vous amener à expliquer et comprendre les phénomènes transitoires de circuits réactifs (à l'aide d'une onde carré à basse fréquence).
Mesurer les comportements des dipôles réactifs soumis à certaines sollicitations. 
Cette séance nous permet également de tenir compte des imperfections des dipôles que nous utilisons par rapport au cas idéal que nous traitons dans les exercices.

\section{Pré-requis}
Avant la séance, vous aurez lu attentivement l'énoncé de la manipulation. Vous aurez par ailleurs relu les chapitres et sections suivants:

\begin{itemize}
	\item Chapitre 2 - Dipôles idéaux
	\begin{itemize}
	    \item Section 2.2 - Dipôles réels >< Dipôles idéaux
	    \item Section 2.3 - Charges idéales : trois effets physiques
	    \item Section 2.6 - Sources idéales
	\end{itemize}
	\item Chapitre 5 - Résoudre un circuit réactif dans le domaine temporel
	\item Chapitre 6 - Résoudre un circuit réactif dans le domaine temporel
	\begin{itemize}
	    \item Section 6.1 - Éléments réactifs : Rappels 
		\item Section 6.2 - Analyse temporelle du circuit RC (source en échelons)
		\item Section 6.3 - Analyse temporelle du circuit RL (source en échelons)
		\item Section 6.5 - Analyse temporelle du circuit RL (source sinusoïdale avec échelon)
		\item Section 6.6 - Analyse temporelle du circuit RLC
	\end{itemize}

\end{itemize}

\vspace{5pt}

\newpage
\section{Préliminaire théorique}
Soit le circuit ci-dessous alimenté par une source de tension continue $V_S$ qui délivre un courant $I$. En tenant compte de l'existence d'un interrupteur représenté par $T_0$ qui sera fermé au temps $t_0 = 0s$, de $R$ une résistance et $L$ une inductance tous deux de valeurs quelconques :
\begin{center}
\begin{circuitikz} \draw
    (0,0)	to[battery1, v=$V_S$, invert, i^>=$I$]		
    (0,4)	to[closing switch, l=$T_0$]		
    (2,4)	to[R, l=$R$]					
    (4,4)	to[L, l=$L$, v_<=$v_L$]					(4,0)--(0,0) node[ground]{}
;
\end{circuitikz}
\end{center}
\Question{
\newline
\begin{enumerate}
    \item Déterminez l'expression analytique de $v_L(t)$ pour tous temps $t\in [T_{-\infty}; T_1]$ avec $T_1 >> T_0$.
    \item Déterminez la constante de temps $\tau$ pour ce circuit?
    \item Comment affecterai l'ajout d'inductances en séries sur la constante de temps $\tau$?
\end{enumerate}
}
{%C
}

\newpage
\section{Partie pratique}
\subsection{Réflexion analytique et physique}
Nous désirons observer des phénomènes transitoires. 
\Question{
\newline
\begin{enumerate}
    \item Quels types de dipôles sont strictement nécessaire pour observer de tels phénomènes? Justifiez à l'aide de la loi du/des dipôle(s) cette affirmation.
    \item Sans interrupteur, comment pourrions-nous observer un effet similaire au régime transitoire pour le circuit suivant :
    \begin{center}
    \begin{circuitikz} \draw
        (0,0)	to[battery1, v=$V_S$, invert, i^>=$I$]		
        (0,4)	to[closing switch, l=$T_0$]		
        (2,4)	to[R, l=$R$]					
        (4,4)	to[L, l=$L$, v_<=$v_L$]				(4,0)--(0,0) node[ground]{}
        ;
    \end{circuitikz}
    \end{center}
\end{enumerate}
}
{%C
}
\newpage
\subsection{Réflexion analytique et physique - Circuit RL}
Considérons le circuit suivant :
\begin{center}
\begin{circuitikz} \draw
        (0,0)	to[square voltage source, v=$v_G(t)$, i^>=$i(t)$]		
        (0,4)	to[R, l=$R_G$]
        (4,4)	to[L, l=$L$]	
        (4,0)   to[R, l=$R_e$] 
        (0,0)   node[ground]{}
        ;
\end{circuitikz}
\end{center}
\Question{
\newline
\begin{enumerate}
    \item Après combien de temps le courant dans le circuit aura atteint 95\% de sa valeur maximale?
    \item Comment régleriez-vous $v_G(t)$ pour que le régime permanent ne s'établisse jamais dans ce circuit? 
\end{enumerate}
}
{%C
}
\newpage
\subsection{Mesures de signaux et interprétation - circuit RL}
Réalisez le circuit suivant sur votre ProtoBoard :
\begin{center}
\begin{circuitikz} \draw
        (0,0)	to[voltage source, v=$v_G(t)$, i^>=$i(t)$]		
        (0,4)	to[R, l=$R_G$]					
        (4,4)	to[L, l=$L$, v_<=$v_L$]	
        (4,0)   to[R, l=$R_e$] 
        (0,0)   node[ground]{}
        ;
\end{circuitikz}
\end{center}
Avec :
\begin{itemize}
    \item $v_G(t)$ une source de tension 
    \item $R_G = 600\Omega$ la résistance de sortie du générateur du PicoScope
    \item $L = 220$µH l'inductance de votre kit de laboratoire
    \item $R_e = 200\Omega$ la résistance étalon
\end{itemize}
\Question{
\newline
\begin{enumerate}
    \item Paramétrez la source de tension de votre générateur pour obtenir une situation similaire à une source de tension continue suivie d'un interrupteur.
    \item Mesurez le temps nécessaire à l'atténuation du régime transitoire de ce circuit.
    \item Comparez cette valeur avec le constante de temps du circuit. Quelle conclusion pouvez-vous en tirer?
    \item Pour une source de fréquence deux fois moins importante, pensez-vous que constante de temps aura une valeur plus basse? Justifiez analytiquement.
\end{enumerate}

}
{%C
}
\newpage
\subsection{Réflexion analytique et physique - Circuit RC}
Considérons le circuit suivant :
\begin{center}
\begin{circuitikz} \draw
        (0,0)	to[square voltage source, v=$v_G(t)$, i^>=$i(t)$]		
        (0,4)	to[R, l=$R_G$]
        (4,4)	to[C, l=$C_i$]	
        (4,0)   to[R, l=$R_e$] 
        (0,0)   node[ground]{}
        ;
\end{circuitikz}
\end{center}
\Question{
\newline
\begin{enumerate}
    \item Après combien de temps le courant dans le circuit aura atteint 63\% de sa valeur maximale?
    \item Comment régleriez-vous $v_G(t)$ pour que le régime permanent ne s'établisse jamais dans ce circuit? 
\end{enumerate}
}
{%C
}
\newpage
\subsection{Mesures de signaux et interprétation - circuit RC}
Réalisez le circuit suivant sur votre ProtoBoard :
\begin{center}
\begin{circuitikz} \draw
        (0,0)	to[voltage source, v=$v_G(t)$, i^>=$i(t)$]		
        (0,4)	to[R, l=$R_G$]					
        (4,4)	to[C, l=$C_i$, v_<=$v_C$]	
        (4,0)   to[R, l=$R_e$] 
        (0,0)   node[ground]{}
        ;
\end{circuitikz}
\end{center}
Avec :
\begin{itemize}
    \item $v_G(t)$ une source de tension 
    \item $R_G = 600\Omega$ la résistance de sortie du générateur du PicoScope
    \item $C_i = 0,1$ µF le condensateur de votre kit de laboratoire
    \item $R_e = 200\Omega$ la résistance étalon
\end{itemize}
\Question{
\newline
\begin{enumerate}
    \item Paramétrez la source de tension de votre générateur pour obtenir une situation similaire à une source de tension continue suivie d'un interrupteur.
    \item Mesurez le temps nécessaire à l'atténuation du régime transitoire de ce circuit.
    \item Comparez cette valeur avec le constante de temps du circuit. Quelle conclusion pouvez-vous en tirer?
    \item Si la source de tension $v_G(t)$ a une amplitude deux fois plus importante, pensez-vous que cela augmentera la valeur de la constante de temps? Justifiez analytiquement.
\end{enumerate}
}
{%C
}

\end{document}