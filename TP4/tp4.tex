%% fancy header & foot
\pagestyle{fancy}
\lhead{[ELEC-H-2001] Électricité\\ LABO \no 3 : Théorème de Thévenin et adaptation d'impédance \ifthenelse{\boolean{corrige}}{~-- corrigé}{}}
\rhead{v1.0.0\\ page \thepage}
\cfoot{}
%%

\pdfinfo{
/Author (Renaud Theunissen et Youssef Agram, ULB -- BEAMS-EE)
/Title (Laboratoire 3 ELEC-H-2001, Théorème de Thévenin et adaptation d'impédance)
/ModDate (D:\pdfdate)
}

\hypersetup{
pdftitle={Labo 3 [ELEC-H-2001] Électricité : LABO 3 ELEC-H-2001, Théorème de Thévenin et adaptation d'impédance},
pdfauthor={Renaud Theunissen et Youssef Agram, ©2020 ULB - BEAMS-EE},}

\setlength{\parskip}{0.5cm plus4mm minus3mm} %espacement entre §
\setlength{\parindent}{0pt}


\begin{document}
\long\def\nothx/*#1*/{}
\tptitle{}{Séance 3~: Théorème de Thévenin et adaptation d'impédance}
\section{But de la manipulation}
L'objectif de cette dernière séance de laboratoire est de vous permettre d'aborder les équivalents de Thévenin et d'en tester l'efficacité ainsi que les limites.
Vous serez également amené à utiliser à nouveau les critères d'adaptation d'impédance pour considérer les charges que nous pourrons ou non brancher aux circuits proposés.

\section{Pré-requis}
Avant la séance, vous aurez lu attentivement l'énoncé de la manipulation. Vous aurez par ailleurs relu les chapitres et sections suivants:
\begin{itemize}
	\item Chapitre 4 - Équivalence de Thévenin et adaptation d'impédance
		\begin{itemize}
		\item Section 4.1 - Circuit équivalents et théorèmes de Thévenin et Norton
		\item Section 4.2 - Impédance d'entrée, impédance de sortie (et fem à vide)
		\item Section 4.3 - Adaptation d'impédance
		\end{itemize}
	\item Chapitre 5 - Résoudre un circuit : procédure de base et accélérateurs
		\begin{itemize}
		\item Section 5.1.2 - Connexions série et parallèle
			\begin{itemize}
			\item Section 5.4 Illustration : Diviseur résistif (étendu aux impédances) 
			\item Section 5.5 Équivalences série et parallèle
			\item Section 5.6 Utilisation des théorèmes de Thévenin et Norton pour résoudre un circuit
			\end{itemize}
		\end{itemize}
	\item Chapitre 7 - Résoudre un circuit réactif dans le domaine fréquentiel
		\begin{itemize}
		\item Section 7.3 - Phaseurs
		\item Section 7.4 - Impédances et admittances
		\end{itemize}
\end{itemize}

\vspace{5pt}

\newpage

\section{Préliminaire Théorique}
\subsection{Équivalent de Thévenin et théorème de superposition}

Soit le circuit suivant :

\begin{center}
\begin{circuitikz} \draw
    (0,0)	to[battery1, v=$V_1$, invert, i^>=$I$]		
    (0,5)	to[R, l=$R$]		
    (2,5)	-- (2,4)--
    (1,4)   to[R, l=$R$]
    (1,2)--(3,2)
            to[R, l_=$R$] (3,4)--(2,4)--(2,5)--
    (5,5)   to[battery1, v=$V_2$, invert] 
    (5,3)   to[R, l_=$R$] 
    (5,1)--(4,1)
        to[R, l=$R$] (2,1)--(2,0)
        to[R, l_=$R$] (4,0)--(4,1)
    (2,1)   to[R, l^=$R$] (0,1)--
    (0,0) node[ground]{}
    (2,2)--(2,1)
    (2,4) to[open, v^<=$V_3$] (2,2)
;
\end{circuitikz}
\end{center}
\Question{
\newline
\begin{itemize}
    \item Déterminez l'équivalent de Thévenin ($V_{Th}$ et $R_{Th}$) de ce circuit vu par la source $V_1$ en fonction de $R$, $V_2$ et $I$.
    \item Déterminez la tension $V_3$ à l'aide du théorème de superposition.
\end{itemize}
}
{%C
}
\newpage
\section{Partie pratique}
\subsection{Réflexion analytique et physique - Circuit résistif}
Considérons le circuit suivant :
\begin{center}
\begin{circuitikz} \draw
(0,0)   to[sinusoidal voltage source, v>=$v_G(t)$, invert, i^>=$i(t)$] 	(0,6)
		to[R, l=$R_G$] (2,6)
		to[R, l=$R$] (4,6)
		(2,6)--(2,7)
		to[R, l=$R$] (4,7)--(4,5)
		to[R, l_=$R$] (2,5)--(2,6)
		(4,6)--(5,6)--(5,7)
		to[R, l=$R$] (7,7)--(7,5)
		to[R, l_=$R$] (5,5)--(5,6)
		(7,6)--(8,6)
		to[generic, l=$R_{charge}$](8,0)--(0,0)
		node[ground]{}	
		node[ocirc] (A) at (8,6) {A}
        node[ocirc] (B) at (8,0) {B}
;
\end{circuitikz}
\end{center}
\Question
{%Question
\newline
\begin{enumerate}
    \item Déterminez l'équivalent de Thévenin vu depuis $R_{charge}$ (entre les point A et B).
    \item Pensez-vous que le type de source peut influencer l'équivalent de Thévenin de ce circuit?
    \item Quelle devra être la valeur de $R_{charge}$ pour que la tension à ces bornes soit égale à 25\% de celle de $v_G$?
\end{enumerate}
}
{%C
}

\newpage
\subsection{Mesures de signaux et interprétation - Circuit résistif}

Soit le circuit suivant avec :

\begin{itemize}
    \item $R_x$ qui est la mise en série de la résistance $R_{x_1}$ et $R_{x_2}$.
    \item $R_i$ et $R_e$ qui ont été déterminées dans les séances précédentes.
    \item La source de tension $v_G(t)$ est une source de 2V d'amplitude et de forme sinusoïde (la pulsation importe peu).
\end{itemize}
\begin{center}
\begin{circuitikz} \draw
(0,0)   to[sinusoidal voltage source, v>=$v_G(t)$, invert, i^>=$i(t)$] 	(0,6)
		to[R, l=$R_G$] (2,6)
		to[R, l=$R_i$] (4,6)
		(2,6)--(2,7)
		to[R, l=$R_x$] (4,7)--(4,5)
		to[R, l_=$R_e$] (2,5)--(2,6)
		(4,6)--(5,6)--(5,7)
		to[R, l=$R_i$] (7,7)--(7,5)
		to[R, l_=$R_i$] (5,5)--(5,6)
		(7,6)--(8,6)
		to[generic, l=$R_{charge}$](8,0)--(0,0)
		node[ground]{}	
		node[ocirc] (A) at (8,6) {A}
        node[ocirc] (B) at (8,0) {B}
;
\end{circuitikz}
\end{center}

\Question {
\newline
\begin{enumerate}
    \item Proposez une procédure pour déterminer la résistance $R_x$ et donnez sa valeur.
    \item En utilisant les valeurs des résistances $R_i$ et $R_e$ trouvées précédemment ainsi que celle de $R_x$, déterminez l'équivalent de Thévenin du circuit proposé.
    \item Quelle sera la tension appliquée sur $R_{charge}$ si cette dernière est égale à $200\Omega$ (donc $R_{charge}$=$R_e$)?
    \item Proposez et tester un montage équivalent à ce dernier mais en utilisant qu'une seule résistance en plus de $R_G$.
\end{enumerate}
}
{%C
}
\newpage
\subsection{Réflexion analytique et physique - Circuit réactif}
Soit le circuit suivant avec:
\begin{itemize}
    \item $R_G = 600\Omega$ la résistance de sortie du générateur.
    \item $C_i = 0,1$µF, $C_j = 22$nF et $C_k = 4,7$nF.
    \item La source de tension $v_G(t)$ est une source de 2V d'amplitude et de forme sinusoïde (la pulsation importe peu).
\end{itemize}
\begin{center}
\begin{circuitikz} \draw
(0,0)   to[sV, v>=$v_G(t)$] 	(0,4)
		to[R, l_=$R_G$] (2,4)--(3,4)--(3,5)
		to[C, l_=$C_i$] (5,5)--(5,4)
(3,4)--(3,3)
        to[C, l_=$C_k$] (5,3)--(5,4)
        to[C, l_=$C_j$] (7,4)--(8,4)
        to[generic, l=$R_{Charge}$] (8,0)--(0,0)
        node[ground]{}
        node[ocirc] (A) at (8,4) {A}
        node[ocirc] (B) at (8,0) {B}
;
\end{circuitikz}
\end{center}
\Question
{%Question
\newline
\begin{enumerate}
    \item Déterminez l'équivalent de Thévenin vu depuis $R_{Charge}$ (entre les point A et B).
    \item Si $R_{Charge}$ est égale à $R_e$, calculez sa différence de potentiel $V_{R_e}$ à partir de votre équivalent de Thévenin.
    \item Si nous plaçons une inductance à la place de la charge, pour quelles fréquences (hautes ou basses) aurons nous une $v_{Charge}$ maximale? Justifiez physiquement.
\end{enumerate}
}
{%C
}

%Pour l'année prochaine celle-ci!
\newpage
\subsection{Mesures de signaux et interprétation - Circuit réactif}
Soit le circuit suivant avec :
\begin{itemize}
    \item $R_G = 600\Omega$ la résistance de sortie du générateur.
    \item $C_i = 0,1$µF, $C_j = 22$nF et $C_k = 4,7$nF.
    \item La source de tension $v_G(t)$ est une source de 2V d'amplitude et de forme sinusoïde à une fréquence de $100$ kHz.
\end{itemize}
\begin{center}
\begin{circuitikz} \draw
(0,0)   to[sV, v>=$v_G(t)$] 	(0,4)
		to[R, l_=$R_G$] (2,4)--(3,4)--(3,5)
		to[C, l_=$C_i$] (5,5)--(5,4)
(3,4)--(3,3)
        to[C, l_=$C_k$] (5,3)--(5,4)
        to[C, l_=$C_j$] (7,4)--(8,4)
        to[generic, l=$R_e$] (8,0)--(0,0)
        node[ground]{}
;
\end{circuitikz}
\end{center}


\Question
{%Question
\newline
\begin{enumerate}
    \item Vérifiez qu'avec l'équivalent de Thévenin que vous avez déterminé précédemment, vous mesurez une différence de potentiel $V_{R_e}$ attendue.
    \item Obtiendriez-vous la même valeur si les condensateurs $C_i$ et $C_j$ étaient permutés comme proposé ci-dessous?
    \begin{center}
\begin{circuitikz} \draw
(0,0)   to[sV, v>=$v_G(t)$] 	(0,4)
		to[R, l_=$R_G$] (2,4)--(3,4)--(3,5)
		to[C, l_=$C_j$] (5,5)--(5,4)
(3,4)--(3,3)
        to[C, l_=$C_k$] (5,3)--(5,4)
        to[C, l_=$C_i$] (7,4)--(8,4)
        to[generic, l=$R_e$] (8,0)--(0,0)
        node[ground]{}
;
\end{circuitikz}
\end{center}
    \item En utilisant une fréquence de $50$kHz pour la source, quelle valeur obtenez-vous pour $V_{R_e}$? Justifiez en discutant la validité de l'équivalent de Thévenin entre ce cas et le précédent.
    \item Pour quelle fréquence obtiendriez-vous la même valeur pour $V_{R_e}$ que dans le cas précédent? 
\end{enumerate}
}
{%C
}

\end{document}